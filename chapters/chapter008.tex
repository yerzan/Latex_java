\chapter{Podsumowanie i wnioski}

Celem niniejszego projektu było zaprojektowanie i zaimplementowanie funkcjonalnego systemu rezerwacji sal oraz zarządzania harmonogramem zajęć dla instytucji edukacyjnej, przy wykorzystaniu języka Java, biblioteki Swing do tworzenia graficznego interfejsu użytkownika, oraz bazy danych PostgreSQL jako backendu do przechowywania danych. Projekt miał na celu połączenie teorii poznanej w trakcie studiów z praktycznym zastosowaniem w formie kompletnej aplikacji.

\section{Osiągnięte rezultaty}

W wyniku realizacji projektu udało się z powodzeniem zaimplementować następujące funkcjonalności:

\begin{itemize}
    \item Stworzono graficzny interfejs użytkownika umożliwiający intuicyjne zarządzanie danymi o zajęciach;
    \item Zaimplementowano możliwość dodawania, edytowania oraz usuwania informacji o zajęciach bezpośrednio z poziomu GUI;
    \item Wprowadzono zaawansowane filtrowanie według sali, grupy oraz typu zajęć;
    \item Zapewniono walidację danych wejściowych oraz obsługę najczęstszych wyjątków logicznych i technicznych;
    \item Stworzono strukturę kodu zgodną z podejściem obiektowym, z wyodrębnieniem warstwy dostępu do danych (DAO), modeli danych oraz interfejsów graficznych;
    \item Połączono aplikację z relacyjną bazą danych PostgreSQL za pomocą technologii JDBC.
\end{itemize}

Projekt został wykonany zgodnie z założeniami oraz przyjętym harmonogramem. Aplikacja przeszła serię testów funkcjonalnych, które potwierdziły poprawność działania wszystkich podstawowych funkcji systemu.

\section{Wnioski}

W trakcie realizacji projektu zdobyto cenne doświadczenie w zakresie projektowania i implementacji aplikacji desktopowych. W szczególności pogłębiono umiejętności w następujących obszarach:

\begin{itemize}
    \item stosowanie zasad programowania obiektowego w praktycznych projektach;
    \item projektowanie i realizacja graficznych interfejsów użytkownika z użyciem Java Swing;
    \item budowa aplikacji z dostępem do relacyjnej bazy danych z użyciem JDBC oraz języka SQL;
    \item stosowanie walidacji danych oraz obsługi wyjątków na poziomie interfejsu i logiki aplikacyjnej;
    \item organizacja kodu w warstwach oraz stosowanie wzorców projektowych takich jak DAO;
    \item testowanie i debugowanie aplikacji desktopowych.
\end{itemize}

System spełnia wszystkie założenia funkcjonalne i stanowi solidną podstawę do dalszego rozwoju. W przyszłości możliwe jest rozszerzenie systemu o dodatkowe funkcjonalności, takie jak:

\begin{itemize}
    \item mechanizm rejestracji i logowania użytkowników z różnymi poziomami dostępu;
    \item eksport danych do formatu PDF lub CSV;
    \item integracja z zewnętrznymi systemami kalendarzowymi (np. Google Calendar);
    \item rozwinięcie systemu o obsługę powiadomień e-mail dla prowadzących i studentów;
    \item przekształcenie projektu w aplikację webową z użyciem nowoczesnych frameworków.
\end{itemize}

Zrealizowany projekt potwierdza możliwość tworzenia złożonych systemów informatycznych nawet w ramach ograniczonego czasu i zasobów, pod warunkiem zastosowania właściwych praktyk inżynierii oprogramowania.

\section{Linki do repozytoriów GitHub}

Źródła projektu oraz dokumentacji dostępne są w publicznych repozytoriach GitHub:

\begin{itemize}
    \item Link do projektu: \url{https://github.com/yerzan/FINALY_PROJECT}
    \item Link do dokumentacji \LaTeX{}: \url{https://github.com/yerzan/Latex_java}
\end{itemize}
