\chapter{Streszczenie}

\section{Streszczenie w języku polskim}

Celem niniejszego projektu było stworzenie aplikacji wspomagającej proces \textbf{rezerwacji sal} oraz zarządzania \textbf{harmonogramem zajęć} na uczelni. Projekt został zrealizowany w języku \textbf{Java} z wykorzystaniem \textbf{graficznego interfejsu użytkownika} (Swing), połączenia z relacyjną bazą danych \textbf{PostgreSQL} za pomocą \textbf{JDBC} oraz wzorca \textbf{DAO} do komunikacji z bazą.

Aplikacja umożliwia wykonywanie operacji \textbf{dodawania}, \textbf{edytowania}, \textbf{filtrowania} i \textbf{usuwania zajęć} dla różnych typów (\textbf{wykład}, \textbf{laboratorium}, \textbf{projekt}). Uwzględnia różnice w obsłudze \textbf{grup} w zależności od typu zajęć oraz zawiera \textbf{walidację danych wejściowych} i odpowiednie \textbf{komunikaty błędów}.

Projekt pozwolił na rozwinięcie umiejętności z zakresu \textbf{programowania obiektowego}, pracy z \textbf{bazami danych} oraz tworzenia aplikacji desktopowych z wykorzystaniem bibliotek \textbf{GUI}.

\section{Abstract in English}

The aim of this project was to create an application that supports the process of \textbf{room reservation} and \textbf{class schedule management} at a university. The project was implemented in \textbf{Java} using a \textbf{graphical user interface} (Swing), connection to a relational database \textbf{PostgreSQL} via \textbf{JDBC}, and the \textbf{DAO design pattern} for data access.

The application allows \textbf{adding}, \textbf{editing}, \textbf{filtering}, and \textbf{deleting classes} of various types (\textbf{lecture}, \textbf{laboratory}, \textbf{project}). It handles \textbf{group selection logic} depending on the type of class and includes \textbf{input validation} with appropriate \textbf{error messages}.

The project enabled the development of skills in \textbf{object-oriented programming}, working with \textbf{databases}, and creating desktop applications using \textbf{GUI libraries}.
