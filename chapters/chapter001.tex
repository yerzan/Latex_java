\section{Wstęp}

\subsection{Cel projektu}

Celem projektu było zaprojektowanie i zaimplementowanie aplikacji desktopowej wspomagającej zarządzanie zajęciami akademickimi oraz rezerwację sal dydaktycznych w środowisku uczelni wyższej.  
System ma za zadanie ułatwić pracownikom sekretariatu organizację planów zajęć, eliminując problemy związane z ręcznym układaniem grafiku.

\subsection{Zakres funkcjonalny systemu}

Dzięki zastosowanym funkcjonalnościom możliwe jest m.in.:
\begin{itemize}
    \item dodawanie nowych zajęć do bazy danych,
    \item filtrowanie według różnych kryteriów (typ, sala, grupa),
    \item edytowanie istniejących rekordów,
    \item sprawdzanie dostępności sal i konfliktów czasowych.
\end{itemize}

Aplikacja automatycznie weryfikuje poprawność danych wejściowych oraz wyświetla komunikaty błędów w przypadku wykrycia kolizji.

\subsection{Zastosowane technologie}

Projekt został zrealizowany w języku \textbf{Java}, przy użyciu biblioteki \textbf{Swing} do stworzenia graficznego interfejsu użytkownika.  
Do komunikacji z bazą danych zastosowano technologię \textbf{JDBC}, natomiast dane przechowywane są w relacyjnej bazie danych \textbf{PostgreSQL}.

\subsection{Struktura dokumentacji}

Dokumentacja została przygotowana z użyciem systemu składu tekstu \LaTeX{} i zawiera:
\begin{itemize}
    \item opis struktury aplikacji,
    \item diagramy klas i bazy danych,
    \item opis interfejsu użytkownika,
    \item opis komponentów i funkcjonalności systemu.
\end{itemize}
