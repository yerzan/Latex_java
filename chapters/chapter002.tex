\chapter{Opis projektu}

\section{Cel i przeznaczenie}

Projekt \textbf{System rezerwacji sal / podział godzin} został zaprojektowany z myślą o ułatwieniu zarządzania harmonogramem zajęć akademickich.  
Głównym celem systemu jest wsparcie administracji uczelni w planowaniu i koordynowaniu zajęć dydaktycznych w sposób zautomatyzowany i intuicyjny.

\section{Główne funkcjonalności}

System umożliwia użytkownikowi wykonywanie następujących operacji:

\begin{itemize}
    \item dodawanie zajęć wraz z informacjami: dzień tygodnia, godzina, typ zajęć, kierunek, przedmiot, prowadzący, sala, grupa;
    \item edytowanie oraz usuwanie wcześniej dodanych zajęć;
    \item filtrowanie zajęć według sali, grupy i typu zajęć;
    \item walidację danych przy wprowadzaniu (np. sprawdzanie konfliktów sal i grup);
    \item obsługę wyjątków i prezentację komunikatów błędów.
\end{itemize}

\section{Typy zajęć}

System rozróżnia trzy typy zajęć, które różnią się liczbą przypisanych grup:

\begin{itemize}
    \item \textbf{Wykład (Wyklad)} – przeznaczony dla wszystkich grup;
    \item \textbf{Projekt (ćwiczenia)} – przeznaczony dla dwóch konkretnych grup (np. A i B);
    \item \textbf{Laboratorium} – przeznaczone dla jednej grupy.
\end{itemize}

\section{Zastosowane technologie}

Do realizacji projektu wykorzystano następujące technologie:

\begin{itemize}
    \item język \textbf{Java};
    \item biblioteka \textbf{Swing} do budowy graficznego interfejsu użytkownika;
    \item baza danych \textbf{PostgreSQL};
    \item interfejs \textbf{JDBC} do komunikacji z bazą danych;
    \item komponent \textbf{LGoodDatePicker} do wyboru daty \cite{lgooddatepicker}.
\end{itemize}
