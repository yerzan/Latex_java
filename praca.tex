\documentclass{urdpl}     % praca w języku polskim

% Lista wszystkich języków stanowiących języki pozycji bibliograficznych użytych w pracy.
% (Zgodnie z zasadami tworzenia bibliografii każda pozycja powinna zostać utworzona zgodnie z zasadami języka, w którym dana publikacja została napisana.)
\usepackage[english,polish]{babel}

% Użyj polskiego łamania wyrazów (zamiast domyślnego angielskiego).
\usepackage{polski}

\usepackage[utf8]{inputenc}

% dodatkowe pakiety

\usepackage{pdfpages}
\usepackage{mathtools}
\usepackage{amsfonts}
\usepackage{amsmath}
\usepackage{amsthm}
\usepackage[hidelinks]{hyperref}
\usepackage{float}
\usepackage{listings}
\usepackage{graphicx}
\usepackage{subcaption}
\usepackage{booktabs}
\usepackage{multirow} 
\usepackage{tabularx} 
\usepackage{amssymb} 
\usepackage{listings}
\usepackage{xcolor}
\usepackage{array}
\usepackage{makecell}
\usepackage[flushleft]{threeparttable}
\usepackage[normalem]{ulem}
\usepackage{lineno}
\usepackage{chngcntr}
% ---------------------------------------------

% --- < bibliografia > ---

\usepackage{csquotes}

% ------------------------
% --- < listingi > ---

% Użyj czcionki kroju Courier.
\usepackage{courier}

\usepackage{listings}
\lstloadlanguages{TeX}
\renewcommand{\lstlistlistingname}{Spis listingów}
\renewcommand{\lstlistingname}{Listing}


\lstset{
	literate={ą}{{\k{a}}}1
           {ć}{{\'c}}1
           {ę}{{\k{e}}}1
           {ó}{{\'o}}1
           {ń}{{\'n}}1
           {ł}{{\l{}}}1
           {ś}{{\'s}}1
           {ź}{{\'z}}1
           {ż}{{\.z}}1
           {Ą}{{\k{A}}}1
           {Ć}{{\'C}}1
           {Ę}{{\k{E}}}1
           {Ó}{{\'O}}1
           {Ń}{{\'N}}1
           {Ł}{{\L{}}}1
           {Ś}{{\'S}}1
           {Ź}{{\'Z}}1
           {Ż}{{\.Z}}1,
	basicstyle=\footnotesize\ttfamily,
}

% defninicja stylu python
\lstdefinestyle{stylePython}{
    language=Python,
    commentstyle=\color{green},          % Kolor komentarzy
    keywordstyle=\color{blue},           % Kolor słów kluczowych
    numberstyle=\tiny\color{gray},       % Kolor i styl numerów linii
    stringstyle=\color{red},             % Kolor ciągów znaków
    basicstyle=\ttfamily\footnotesize,   % Podstawowy styl kodu
    breakatwhitespace=false,             % Automatyczne dzielenie wierszy
    breaklines=true,                     % Dzielenie długich linii
    keepspaces=true,                     % Zachowanie spacji
    numbers=left,                        % Numery linii po lewej
    numbersep=5pt,                       % Odstęp numerów od kodu
    showspaces=false,                    % Nie pokazuj spacji
    showstringspaces=false,              % Nie pokazuj spacji w ciągach znaków
    showtabs=false,                      % Nie pokazuj tabulacji
    tabsize=2                            % Rozmiar tabulacji
}

% defnicja stylu JAVA
\lstdefinestyle{javaStyle}{
    language=Java,
    basicstyle=\ttfamily\footnotesize,
    keywordstyle=\color{blue},
    commentstyle=\color{green!50!black}\itshape,
    stringstyle=\color{green},
    numberstyle=\tiny\color{gray},
    numbers=left,
    numbersep=5pt,                       % Odstęp numerów od kodu
    stepnumber=1,
    showspaces=false,                    % Nie pokazuj spacji
    tabsize=2,
    showstringspaces=false,
    breaklines=true,
    breakatwhitespace=false,             % Automatyczne dzielenie wierszy
    showtabs=false,                      % Nie pokazuj tabulacji
    keepspaces=true                    % Zachowanie spacji
}


\definecolor{stringcolor}{RGB}{163,21,21}    % pomarańczowy - stringi
\definecolor{typecolor}{RGB}{43, 145, 176}     % ciemny fiolet - klasy, typy

\lstdefinestyle{csStyle}{
    language=[Sharp]C, % dla C#; można zmienić na Java
    basicstyle=\ttfamily\footnotesize,
    keywordstyle=\color{blue},
    stringstyle=\color{stringcolor},
    commentstyle=\color{green!50!black}\itshape,
    morekeywords={class, public, private, protected, static, void, string, int, new}, % dodatkowe słowa kluczowe
    emphstyle=\color{typecolor}\bfseries, % klasy na fioletowo
    numbers=left,
    numbersep=5pt,                       % Odstęp numerów od kodu
    numberstyle=\tiny\color{gray},
    stepnumber=1,
    breaklines=true,
    showspaces=false,                    % Nie pokazuj spacji
    tabsize=2,
    showstringspaces=false,
    breakatwhitespace=false,             % Automatyczne dzielenie wierszy
    showtabs=false,                      % Nie pokazuj tabulacji
    keepspaces=true                    % Zachowanie spacji  
}

\definecolor{lightgray}{rgb}{0.9,0.9,0.9}
    % \definecolor{blue}{rgb}{0,0,1}
    \definecolor{green}{rgb}{0,0.6,0}
    % \definecolor{red}{rgb}{0.6,0,0}
    \definecolor{gray}{rgb}{0.5,0.5,0.5}


\AtBeginDocument{
  \counterwithout{figure}{chapter}
  \counterwithout{table}{chapter}
  \counterwithout{lstlisting}{chapter}
}

% % ------------------------
\AtBeginDocument{
	\renewcommand{\tablename}{Tabela}
	\renewcommand{\figurename}{Rys.}   
    \newcommand{\listingname}{Listing}
}


% ------------------------
% --- < tabele > ---

% defines the X column to use m (\parbox[c]) instead of p (`parbox[t]`)
\newcolumntype{C}[1]{>{\hsize=#1\hsize\centering\arraybackslash}X}

%---------------------------------------------------------------------------

\author{Mykhailo Kleban}
\shortauthor{M. Kleban}
\noAlbum{134922}

\titlePL{System rezerwacji sal/podział godzin}
\titleEN{Thesis in \LaTeX}

\shorttitlePL{System rezerwacji sal/podział godzin – dokumentacja projektu} % skrócona wersja tytułu jeśli jest bardzo długi
\shorttitleEN{Preparation of a long and fascinating thesis in \LaTeX}

\thesistype{Praca projektowa}


\thesisDone{Praca wykonana pod kierunkiem}
\supervisor{dr inż. Ewa Żesławska}
%\supervisor{Jan Nowak PhD}

\degreeprogramme{Informatyka}
%\degreeprogramme{Computer Science}

\date{2025}

\department{Instytut Informatyki}
%\department{Institute of Computer Science}

\faculty{Wydział Nauk Ścisłych i Technicznych}
%\faculty{Faculty of Science and Technology}



\setlength{\cftsecnumwidth}{10mm}

%---------------------------------------------------------------------------
\setcounter{secnumdepth}{4}
\brokenpenalty=10000\relax


% --------------------------------------------------------------------------
% główna część pracy
% --------------------------------------------------------------------------

\begin{document}

\titlepages

% Ponowne zdefiniowanie stylu `plain`, aby usunąć numer strony z pierwszej strony spisu treści i poszczególnych rozdziałów.
\fancypagestyle{plain}
{
    % Usuń nagłówek i stopkę
    \fancyhf{}
    % Usuń linie.
    \renewcommand{\headrulewidth}{0pt}
    \renewcommand{\footrulewidth}{0pt}
}

\setcounter{tocdepth}{2}
\tableofcontents
\clearpage


% dodanie poszczególnych rozdziałów 

\section{Wstęp}

\subsection{Cel projektu}

Celem projektu było zaprojektowanie i zaimplementowanie aplikacji desktopowej wspomagającej zarządzanie zajęciami akademickimi oraz rezerwację sal dydaktycznych w środowisku uczelni wyższej.  
System ma za zadanie ułatwić pracownikom sekretariatu organizację planów zajęć, eliminując problemy związane z ręcznym układaniem grafiku.

\subsection{Zakres funkcjonalny systemu}

Dzięki zastosowanym funkcjonalnościom możliwe jest m.in.:
\begin{itemize}
    \item dodawanie nowych zajęć do bazy danych,
    \item filtrowanie według różnych kryteriów (typ, sala, grupa),
    \item edytowanie istniejących rekordów,
    \item sprawdzanie dostępności sal i konfliktów czasowych.
\end{itemize}

Aplikacja automatycznie weryfikuje poprawność danych wejściowych oraz wyświetla komunikaty błędów w przypadku wykrycia kolizji.

\subsection{Zastosowane technologie}

Projekt został zrealizowany w języku \textbf{Java}, przy użyciu biblioteki \textbf{Swing} do stworzenia graficznego interfejsu użytkownika.  
Do komunikacji z bazą danych zastosowano technologię \textbf{JDBC}, natomiast dane przechowywane są w relacyjnej bazie danych \textbf{PostgreSQL}.

\subsection{Struktura dokumentacji}

Dokumentacja została przygotowana z użyciem systemu składu tekstu \LaTeX{} i zawiera:
\begin{itemize}
    \item opis struktury aplikacji,
    \item diagramy klas i bazy danych,
    \item opis interfejsu użytkownika,
    \item opis komponentów i funkcjonalności systemu.
\end{itemize}

\chapter{Opis projektu}

\section{Cel i przeznaczenie}

Projekt \textbf{System rezerwacji sal / podział godzin} został zaprojektowany z myślą o ułatwieniu zarządzania harmonogramem zajęć akademickich.  
Głównym celem systemu jest wsparcie administracji uczelni w planowaniu i koordynowaniu zajęć dydaktycznych w sposób zautomatyzowany i intuicyjny.

\section{Główne funkcjonalności}

System umożliwia użytkownikowi wykonywanie następujących operacji:

\begin{itemize}
    \item dodawanie zajęć wraz z informacjami: dzień tygodnia, godzina, typ zajęć, kierunek, przedmiot, prowadzący, sala, grupa;
    \item edytowanie oraz usuwanie wcześniej dodanych zajęć;
    \item filtrowanie zajęć według sali, grupy i typu zajęć;
    \item walidację danych przy wprowadzaniu (np. sprawdzanie konfliktów sal i grup);
    \item obsługę wyjątków i prezentację komunikatów błędów.
\end{itemize}

\section{Typy zajęć}

System rozróżnia trzy typy zajęć, które różnią się liczbą przypisanych grup:

\begin{itemize}
    \item \textbf{Wykład (Wyklad)} – przeznaczony dla wszystkich grup;
    \item \textbf{Projekt (ćwiczenia)} – przeznaczony dla dwóch konkretnych grup (np. A i B);
    \item \textbf{Laboratorium} – przeznaczone dla jednej grupy.
\end{itemize}

\section{Zastosowane technologie}

Do realizacji projektu wykorzystano następujące technologie:

\begin{itemize}
    \item język \textbf{Java};
    \item biblioteka \textbf{Swing} do budowy graficznego interfejsu użytkownika;
    \item baza danych \textbf{PostgreSQL};
    \item interfejs \textbf{JDBC} do komunikacji z bazą danych;
    \item komponent \textbf{LGoodDatePicker} do wyboru daty \cite{lgooddatepicker}.
\end{itemize}

\chapter{Implementacja systemu}

System został zaimplementowany w języku \textbf{Java}, z wykorzystaniem biblioteki \textbf{Swing} do budowy graficznego interfejsu użytkownika oraz technologii \textbf{JDBC} do komunikacji z bazą danych \textbf{PostgreSQL}.

\section{Struktura aplikacji}

Struktura projektu została logicznie podzielona na pakiety zgodnie z zasadami dobrej organizacji kodu. W folderze \texttt{src} znajdują się wszystkie elementy źródłowe aplikacji, zorganizowane w następujący sposób:

\begin{itemize}
    \item \texttt{dao} – klasy odpowiedzialne za dostęp do danych:
    \begin{itemize}
        \item \texttt{LoginDAO} – obsługa uwierzytelniania użytkownika;
        \item \texttt{ZajeciaDAO} – operacje CRUD na tabeli zajęć.
    \end{itemize}

    \item \texttt{database} – logika połączenia z bazą danych:
    \begin{itemize}
        \item \texttt{DatabaseConnection} – klasa łącząca aplikację z PostgreSQL.
    \end{itemize}

    \item \texttt{DodajZajeciaPanel} – komponent GUI odpowiedzialny za dodawanie nowych zajęć:
    \begin{itemize}
        \item \texttt{DodajZajeciaPanel.java/.form} – panel formularza oraz jego widok.
    \end{itemize}

    \item \texttt{EdytujZajeciaPanel} – komponent GUI służący do edycji zajęć:
    \begin{itemize}
        \item \texttt{EdytujZajeciaPanel.java/.form} – logika i widok edycji.
    \end{itemize}

    \item \texttt{Login} – komponent odpowiedzialny za ekran logowania:
    \begin{itemize}
        \item \texttt{Login.java/.form} – widok i obsługa logowania.
    \end{itemize}

    \item \texttt{model} – klasy reprezentujące dane biznesowe:
    \begin{itemize}
        \item \texttt{Zajecia} – klasa bazowa reprezentująca ogólne zajęcia;
        \item \texttt{Wyklad}, \texttt{Projekt}, \texttt{Laboratorium} – klasy dziedziczące;
        \item \texttt{PlanZajec} – klasa pomocnicza reprezentująca pojedynczy wpis w planie.
    \end{itemize}

    \item \texttt{SekretariatPanel} – główny interfejs do zarządzania zajęciami:
    \begin{itemize}
        \item \texttt{SekretariatPanel.java/.form} – widok panelu oraz jego logika.
    \end{itemize}

    \item \texttt{resourse} – folder przechowujący zasoby zewnętrzne (np. ikony lub grafiki).

    \item \texttt{Main.java} – klasa uruchamiająca aplikację.
\end{itemize}

Takie rozdzielenie pozwala na lepszą czytelność kodu, łatwiejsze zarządzanie komponentami oraz zgodność z zasadami programowania obiektowego.

\chapter{Harmonogram realizacji projektu}

W poniższej tabeli oraz na wykresie Gantta przedstawiono plan realizacji projektu wraz z zakładanym czasem trwania każdego etapu.

\begin{figure}[H]
    \centering
    \includegraphics[width=\textwidth]{figures/diagram_gantta_colored_fixed.png}
    \caption{Harmonogram realizacji projektu w formie wykresu Gantta}
    \label{fig:gantt_diagram}
\end{figure}

\chapter{Opis interfejsu uzytkownika}

\section*{Panel logowania}
\begin{figure}[H]
\centering
\includegraphics[width=0.4\textwidth]{figures/workApl/login_panel.png}
\caption{Panel logowania do systemu}
\label{fig:login_panel}
\end{figure}

Panel logowania umożliwia autoryzację użytkownika przed uzyskaniem dostępu do głównego interfejsu systemu. Użytkownik wprowadza dane w pola \texttt{Login} i \texttt{Password}, a następnie klika przycisk \textbf{Zaloguj}. Przy nieprawidłowych danych wyświetlany jest komunikat błędu.

\section*{Główny panel sekretariatu}
\begin{figure}[H]
\centering
\includegraphics[width=\textwidth]{figures/workApl/mainpanel.png}
\caption{Główny panel sekretariatu z listą zajęć i filtrowaniem}
\label{fig:mainpanel}
\end{figure}

Główny panel aplikacji wyświetla listę wszystkich zajęć w formie tabeli. Użytkownik może:
\begin{itemize}
    \item \textbf{Dodawać zajęcia} – klikając przycisk \textbf{Dodaj},
    \item \textbf{Edytować istniejące zajęcia} – po zaznaczeniu wiersza i kliknięciu \textbf{Edytuj},
    \item \textbf{Usuwać zajęcia} – klikając przycisk \textbf{Usuń},
    \item \textbf{Filtrować dane} – za pomocą rozwijanych list: sala, grupa, typ zajęć,
    \item \textbf{Odświeżać widok} – klikając przycisk \textbf{Odśwież}.
\end{itemize}

\section*{Formularz dodawania zajęć typu Wykład}
\begin{figure}[H]
\centering
\includegraphics[width=0.45\textwidth]{figures/workApl/add_wyklad_panel.png}
\caption{Formularz dodawania zajęć typu Wykład}
\label{fig:add_wyklad}
\end{figure}

Użytkownik wprowadza dane dotyczące wykładu: kierunek, przedmiot, prowadzący, sala, dzień tygodnia, godzina. Zajęcia typu wykład są przypisane do wszystkich grup, więc pola \texttt{Grupa 1} i \texttt{Grupa 2} są wyszarzone.

\section*{Formularz dodawania zajęć typu Laboratorium}
\begin{figure}[H]
\centering
\includegraphics[width=0.45\textwidth]{figures/workApl/add_lab_panel.png}
\caption{Formularz dodawania zajęć typu Laboratorium}
\label{fig:add_lab}
\end{figure}

W tym formularzu użytkownik wybiera jedną grupę laboratoryjną, dla której są przeznaczone dane zajęcia. Pole \texttt{Grupa 2} jest wyłączone.

\section*{Formularz dodawania zajęć typu Projekt}
\begin{figure}[H]
\centering
\includegraphics[width=0.45\textwidth]{figures/workApl/add_projekt_panel.png}
\caption{Formularz dodawania zajęć typu Projekt}
\label{fig:add_projekt}
\end{figure}

Dla zajęć typu Projekt dostępne są dwa pola wyboru grup: \texttt{Grupa 1} i \texttt{Grupa 2}. System wymusza wybranie obu, ponieważ projekty realizowane są wspólnie przez dwie grupy.

\section*{Formularz edycji zajęć}
\begin{figure}[H]
\centering
\includegraphics[width=0.45\textwidth]{figures/workApl/edit_panel.png}
\caption{Formularz edycji istniejących zajęć}
\label{fig:edit_panel}
\end{figure}

Formularz edycji pozwala na modyfikację danych wcześniej zapisanych zajęć. Po zaznaczeniu wiersza w tabeli, dane są automatycznie ładowane do formularza. Po kliknięciu \textbf{Zapisz}, rekord zostaje zaktualizowany w bazie danych. System waliduje dane przed zapisem.


\section{Baza danych}

Dane przechowywane są w relacyjnej bazie danych \textbf{PostgreSQL}, w ramach schematu \texttt{public} bazy \texttt{javabase}. System korzysta z następujących tabel:

\begin{itemize}
    \item \texttt{zajecia} – główna tabela przechowująca dane o wszystkich zajęciach:
    \begin{itemize}
        \item \texttt{id}, \texttt{typ}, \texttt{kierunek}, \texttt{przedmiot}, \texttt{prowadzacy}, \texttt{sala}, \texttt{dzien}, \texttt{godzina}, \texttt{grupa1}, \texttt{grupa2}.
    \end{itemize}

    \item \texttt{projekty} – zawiera przypisanie dwóch grup do zajęć typu \texttt{Projekt}:
    \begin{itemize}
        \item \texttt{zajecia\_id}, \texttt{grupa1}, \texttt{grupa2}.
    \end{itemize}

    \item \texttt{laboratoria} – przechowuje numer grupy przypisanej do zajęć typu \texttt{Laboratorium}:
    \begin{itemize}
        \item \texttt{zajecia\_id}, \texttt{nr\_grupy}.
    \end{itemize}

    \item \texttt{uzytkownicy} – tabela logowania przechowująca dane uwierzytelniające użytkowników:
    \begin{itemize}
        \item \texttt{id}, \texttt{login}, \texttt{haslo}.
    \end{itemize}
\end{itemize}

Tabele są powiązane logicznie przez kolumnę \texttt{zajecia\_id}, a dane zabezpieczone są poprzez ograniczenia integralności oraz indeksy. Struktura została zaprojektowana tak, aby umożliwiać wygodne wykonywanie operacji CRUD i filtrowania.

\section{Diagram Baza Danych}

\begin{figure}[H]
    \centering
    \includegraphics[width=0.85\textwidth]{figures/diagramDB.png}
    \caption{Diagram relacyjny bazy danych systemu}
    \label{fig:diagram-db}
\end{figure}
\section{Interfejs użytkownika}

Interfejs został wykonany w technologii \textbf{Swing}. Główne okna aplikacji to:

\begin{itemize}
    \item \textbf{Ekran logowania} – umożliwia dostęp tylko zalogowanym użytkownikom;
    \item \textbf{Panel sekretariatu} – pozwala na przeglądanie i zarządzanie zajęciami;
    \item \textbf{Formularz dodawania zajęć} – umożliwia wprowadzenie nowych zajęć;
    \item \textbf{Formularz edycji zajęć} – pozwala na modyfikację istniejących rekordów.
\end{itemize}

\section{Walidacja i błędy}

System zawiera zabezpieczenia:

\begin{itemize}
    \item Sprawdzanie dostępności sali w wybranym dniu i godzinie;
    \item Sprawdzanie, czy dana grupa nie ma już zajęć w tym czasie;
    \item Obsługa wyjątków SQL i wyświetlanie komunikatów błędów użytkownikowi.
\end{itemize}

\section{Szczegółowy opis komponentów GUI}

\subsection {Formularz Dodawania Zajęć (DodajZajeciaPanel)}

\begin{itemize}
    \item \textbf{Cel:} Dodawanie nowych rekordów zajęć do bazy danych.
    \item \textbf{Elementy:}
    \begin{itemize}
        \item \texttt{JComboBox} – typ zajęć (np. Wykład, Laboratorium, Projekt);
        \item \texttt{JTextField} – kierunek, przedmiot, prowadzący, sala, godzina, grupa1, grupa2;
        \item \texttt{JComboBox} – dzień tygodnia;
        \item \texttt{JButton} – „Zapisz” — zapisuje dane do bazy danych.
    \end{itemize}
    \item \textbf{Uwagi:} Interfejs zawiera etykiety (\texttt{JLabel}) przypisane do każdego pola. Formularz obsługuje walidację przed zapisem zajęć.
\end{itemize}

\subsection {Formularz Edytowania Zajęć (EdytujZajeciaPanel)}

\begin{itemize}
    \item \textbf{Cel:} Edytowanie istniejących danych zajęć.
    \item \textbf{Elementy:} Te same co w \texttt{DodajZajeciaPanel}, jednak służą do aktualizacji danych:
    \begin{itemize}
        \item Pola są wstępnie wypełnione danymi z wybranego rekordu;
        \item \texttt{JButton} – „Zapisz” — aktualizuje dane w bazie.
    \end{itemize}
    \item \textbf{Uwagi:} Pola są edytowalne i automatycznie uzupełniane na podstawie danych wybranych z tabeli.
\end{itemize}

\subsection {Formularz Logowania (Login)}

\begin{itemize}
    \item \textbf{Cel:} Autoryzacja użytkownika w systemie.
    \item \textbf{Elementy:}
    \begin{itemize}
        \item \texttt{JTextField} – login;
        \item \texttt{JPasswordField} – hasło;
        \item \texttt{JButton} – „Zaloguj” — weryfikuje dane logowania.
    \end{itemize}
    \item \textbf{Uwagi:} Prosty, nowoczesny wygląd z ikoną kłódki; Zabezpieczenie dostępu do aplikacji.
\end{itemize}

\subsection {Panel Sekretariatu (SekretariatPanel)}

\begin{itemize}
    \item \textbf{Cel:} Zarządzanie zajęciami (CRUD).
    \item \textbf{Elementy:}
    \begin{itemize}
        \item \texttt{JTable} – wyświetlanie listy zajęć;
        \item \texttt{JComboBox} – filtrowanie po sali, grupie, typie zajęć;
        \item \texttt{JButton} – „Dodaj”, „Usuń”, „Edytuj”, „Zastosuj filtry”, „Odśwież”;
        \item \texttt{JScrollPane} – przewijana tabela;
        \item \texttt{JLabel} – etykiety opisowe.
    \end{itemize}
    \item \textbf{Uwagi:} Obsługuje filtrowanie i pełną obsługę CRUD; główny ekran zarządzania.
\end{itemize}

\section{Diagram komponentów GUI i zależności}
\vspace{1em}
\noindent\textbf{Diagram komponentów GUI i zależności}

\vspace{0.5em}
\begin{figure}[H]
    \centering
    \includegraphics[width=0.95\textwidth]{figures/diagram.png}
    \caption{Zależności pomiędzy komponentami GUI, modelem danych i warstwą DAO}
    \label{fig:diagram-gui}
\end{figure}

\chapter{Testowanie systemu}

System został przetestowany manualnie poprzez interfejs graficzny oraz poprzez analizę zapytań SQL wykonywanych przez klasę \texttt{ZajeciaDAO}. Testy przeprowadzono w środowisku lokalnym z bazą danych \textbf{PostgreSQL}.

\section{Testy logowania}

Przetestowano poprawność działania formularza logowania:

\begin{figure}[H]
\centering
\includegraphics[width=0.4\textwidth]{figures/approve/approve_login.png}
\caption{Logowanie zakończone sukcesem}
\end{figure}

\begin{figure}[H]
\centering
\includegraphics[width=0.4\textwidth]{figures/Errors/bad_login.png}
\caption{Blad logowania – nieprawidlowe dane}
\end{figure}

\section{Testy dodawania zajęć}

Przetestowano dodawanie różnych typów zajęć:

\begin{itemize}
    \item Dodanie zajęć typu \texttt{Projekt} – poprawne wstawienie danych do bazy;
    \item Sprawdzenie walidacji grupy i sali – przy próbie konfliktu system wyświetla stosowny komunikat;
    \item Dodanie zajęć tylko dla jednej grupy (np. \texttt{grupaA}) działa prawidłowo.
\end{itemize}

\begin{figure}[H]
\centering
\includegraphics[width=0.5\textwidth]{figures/approve/add_zajecia_succesfull.png}
\caption{Potwierdzenie dodania zajęć}
\end{figure}

\section{Testy filtrowania danych}

Sprawdzono poprawność działania filtrów:

\begin{itemize}
    \item Filtrowanie po dniu i grupie – poprawne ograniczenie wyników;
    \item Filtrowanie po typie zajęć i przedmiocie – wyświetlane są tylko pasujące rekordy;
    \item Filtrowanie nieistniejących danych – system poprawnie wyświetla pustą tabelę bez błędów.
\end{itemize}

\section{Testy edycji i usuwania zajęć}

Przetestowano operacje modyfikacji i usuwania danych:

\begin{itemize}
    \item Edycja sali i prowadzącego – zmiany są natychmiast widoczne w tabeli GUI;
    \item Usunięcie zajęć z bazy powoduje ich zniknięcie z interfejsu;
    \item Obsługa wyjątków SQL przy próbie edycji nieistniejącego rekordu działa prawidłowo.
\end{itemize}

\begin{figure}[H]
\centering
\includegraphics[width=0.5\textwidth]{figures/Warning/select_zajecia_to_edit.png}
\caption{Brak zaznaczonego rekordu do edycji}
\end{figure}

\begin{figure}[H]
\centering
\includegraphics[width=0.5\textwidth]{figures/Warning/delete_warning.png}
\caption{Potwierdzenie usuniecia zajec}
\end{figure}

\begin{figure}[H]
\centering
\includegraphics[width=0.5\textwidth]{figures/Warning/select_zajecia.png}
\caption{Brak zaznaczonego rekordu do usuniecia}
\end{figure}

\section{Testy walidacji danych}

Sprawdzono poprawność działania walidacji formularza:

\begin{itemize}
    \item Puste pola – system wyświetla komunikat bledu i nie zapisuje danych;
    \item Nieprawidlowe dane (np. litery w polu godzina) – poprawna obsluga bledu i komunikat;
    \item Proba dodania zajec do juz zajetej sali – system blokuje takie dodanie;
    \item Konflikt grupy – uzytkownik otrzymuje odpowiedni alert.
\end{itemize}

\begin{figure}[H]
\centering
\includegraphics[width=0.55\textwidth]{figures/Errors/add_panel_error.png}
\caption{Puste pola formularza – komunikat bledu}
\end{figure}

\begin{lstlisting}[caption={Walidacja pustych pol}, label={lst:puste}]
if (kierunek.isEmpty() || przedmiot.isEmpty() || prowadzacy.isEmpty()) {
    JOptionPane.showMessageDialog(null, "Wszystkie pola tekstowe musza byc wypelnione!",
        "Blad", JOptionPane.WARNING_MESSAGE, warningIcon);
    return;
}
\end{lstlisting}

\begin{figure}[H]
\centering
\includegraphics[width=0.55\textwidth]{figures/Errors/sala_zajeta_error.png}
\caption{Blad – sala juz zajeta w tym terminie}
\end{figure}

\begin{lstlisting}[caption={Sprawdzanie dostepnosci sali}, label={lst:sala}]
if (dao.czySalaZajeta(sala, dzien, godzina, null)) {
    JOptionPane.showMessageDialog(null, "Sala jest juz zajeta w tym dniu i godzinie!",
        "Konflikt", JOptionPane.WARNING_MESSAGE, warningIcon);
    return;
}
\end{lstlisting}

\begin{lstlisting}[caption={Sprawdzenie konfliktu grupy}, label={lst:grupa}]
JOptionPane.showMessageDialog(null, "Wybrana grupa ma juz zajecia w tym dniu i godzinie!",
    "Konflikt", JOptionPane.WARNING_MESSAGE, warningIcon);
return;
\end{lstlisting}

\chapter{Podsumowanie i wnioski}

Celem niniejszego projektu było zaprojektowanie i zaimplementowanie funkcjonalnego systemu rezerwacji sal oraz zarządzania harmonogramem zajęć dla instytucji edukacyjnej, przy wykorzystaniu języka Java, biblioteki Swing do tworzenia graficznego interfejsu użytkownika, oraz bazy danych PostgreSQL jako backendu do przechowywania danych. Projekt miał na celu połączenie teorii poznanej w trakcie studiów z praktycznym zastosowaniem w formie kompletnej aplikacji.

\section{Osiągnięte rezultaty}

W wyniku realizacji projektu udało się z powodzeniem zaimplementować następujące funkcjonalności:

\begin{itemize}
    \item Stworzono graficzny interfejs użytkownika umożliwiający intuicyjne zarządzanie danymi o zajęciach;
    \item Zaimplementowano możliwość dodawania, edytowania oraz usuwania informacji o zajęciach bezpośrednio z poziomu GUI;
    \item Wprowadzono zaawansowane filtrowanie według sali, grupy oraz typu zajęć;
    \item Zapewniono walidację danych wejściowych oraz obsługę najczęstszych wyjątków logicznych i technicznych;
    \item Stworzono strukturę kodu zgodną z podejściem obiektowym, z wyodrębnieniem warstwy dostępu do danych (DAO), modeli danych oraz interfejsów graficznych;
    \item Połączono aplikację z relacyjną bazą danych PostgreSQL za pomocą technologii JDBC.
\end{itemize}

Projekt został wykonany zgodnie z założeniami oraz przyjętym harmonogramem. Aplikacja przeszła serię testów funkcjonalnych, które potwierdziły poprawność działania wszystkich podstawowych funkcji systemu.

\section{Wnioski}

W trakcie realizacji projektu zdobyto cenne doświadczenie w zakresie projektowania i implementacji aplikacji desktopowych. W szczególności pogłębiono umiejętności w następujących obszarach:

\begin{itemize}
    \item stosowanie zasad programowania obiektowego w praktycznych projektach;
    \item projektowanie i realizacja graficznych interfejsów użytkownika z użyciem Java Swing;
    \item budowa aplikacji z dostępem do relacyjnej bazy danych z użyciem JDBC oraz języka SQL;
    \item stosowanie walidacji danych oraz obsługi wyjątków na poziomie interfejsu i logiki aplikacyjnej;
    \item organizacja kodu w warstwach oraz stosowanie wzorców projektowych takich jak DAO;
    \item testowanie i debugowanie aplikacji desktopowych.
\end{itemize}

System spełnia wszystkie założenia funkcjonalne i stanowi solidną podstawę do dalszego rozwoju. W przyszłości możliwe jest rozszerzenie systemu o dodatkowe funkcjonalności, takie jak:

\begin{itemize}
    \item mechanizm rejestracji i logowania użytkowników z różnymi poziomami dostępu;
    \item eksport danych do formatu PDF lub CSV;
    \item integracja z zewnętrznymi systemami kalendarzowymi (np. Google Calendar);
    \item rozwinięcie systemu o obsługę powiadomień e-mail dla prowadzących i studentów;
    \item przekształcenie projektu w aplikację webową z użyciem nowoczesnych frameworków.
\end{itemize}

Zrealizowany projekt potwierdza możliwość tworzenia złożonych systemów informatycznych nawet w ramach ograniczonego czasu i zasobów, pod warunkiem zastosowania właściwych praktyk inżynierii oprogramowania.

\section{Linki do repozytoriów GitHub}

Źródła projektu oraz dokumentacji dostępne są w publicznych repozytoriach GitHub:

\begin{itemize}
    \item Link do projektu: \url{https://github.com/yerzan/PROJECT2025}
    \item Link do dokumentacji \LaTeX{}: \url{https://github.com/yerzan/Latex_java}
\end{itemize}


% Wyłączenie działania `ulem` na czas bibliografii
\renewcommand{\emph}[1]{\textit{#1}}
% Bibliografia
% Dodanie bibliografi do spisu treści
\addcontentsline{toc}{section}{\textbf{Bibliografia}}
\bibliographystyle{plain}
\bibliography{bibliografia}

% Przywrócenie działania `ulem`
\renewcommand{\emph}[1]{\uline{#1}}

\clearpage
% Dodanie spisu rysunków do spisu treści
\addcontentsline{toc}{section}{\textbf{Spis rysunków}}
\listoffigures
\clearpage

% Dodanie spisu tabel do spisu treści
%\addcontentsline{toc}{section}{\textbf{Spis tabel}}
%\listoftables
%\clearpage


\clearpage

% Dodanie spisu listingow do spisu treści
\addcontentsline{toc}{section}{\textbf{Spis listingów}}
\lstlistoflistings
\clearpage


% \appendix
\chapter*{}
\label{cha:statement-A}
\makeatletter
\addcontentsline{toc}{section}{\textbf{Oświadczenie studenta o samodzielności pracy}}

\noindent
\begin{flushright}
    \begin{minipage}[!h]{10cm}
        Załącznik nr 2 do Zarządzenia nr 228/2021 Rektora Uniwersytetu Rzeszowskiego z dnia 1 grudnia 2021 roku w sprawie ustalenia procedury antyplagiatowej w Uniwersytecie Rzeszowskim
    \end{minipage}
\end{flushright}

\begin{center}
    \vspace*{10mm}
    \noindent  {\textbf{OŚWIADCZENIE STUDENTA O SAMODZIELNOŚCI PRACY} }
    \vspace*{10mm}
\end{center}

\noindent
\dotuline{\hspace{1.3cm}\@author\hspace{1.3cm}}\\ % Linia pozioma
{\small Imię (imiona) i nazwisko studenta }\\

\noindent \@faculty\\

\noindent \dotuline{\hspace{1.4cm}\@degreeprogramme \hspace{1.4cm}}\\
{\small Nazwa kierunku} \\

\noindent \dotuline{\hspace{1.8cm}\@noAlbum\hspace{1.9cm}}\\
{\small Numer albumu}

\begin{enumerate}
    \item Oświadczam, że moja praca projektowa pt.: \@titlePL
          \begin{enumerate}[label=\arabic*)]
              \item została przygotowana przeze mnie samodzielnie*,
              \item nie narusza praw autorskich w rozumieniu ustawy z dnia 4 lutego 1994 roku o prawie autorskim i prawach pokrewnych (t.j. Dz.U. z 2021 r., poz. 1062) oraz dóbr osobistych chronionych prawem cywilnym,
              \item nie zawiera danych i informacji, które uzyskałem/am w sposób niedozwolony,
              \item nie była podstawą otrzymania oceny z innego przedmiotu na uczelni wyższej ani mnie, ani innej osobie.
          \end{enumerate}
    \item Jednocześnie wyrażam zgodę/nie wyrażam zgody** na udostępnienie mojej pracy projektowej do celów naukowo--badawczych z poszanowaniem przepisów ustawy o prawie autorskim i prawach pokrewnych.
\end{enumerate}


\vspace*{10mm}

\noindent
\underline{\hspace{6cm}} \hfill \underline{\hspace{6cm}} \\ % Puste miejsce na miejscowość, data oraz podpis
\hspace*{13mm}(miejscowość, data)  \hspace*{63mm}(czytelny podpis studenta)
\vspace*{10mm}

\vfill
\noindent
* Uwzględniając merytoryczny wkład prowadzącego przedmiot \\
** -- niepotrzebne skreślić


\end{document}
